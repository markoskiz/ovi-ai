% !TEX encoding = UTF-8 Unicode
\documentclass{beamer}

\usefonttheme{professionalfonts}

\usepackage{fontspec}
\newfontfamily\greekfontsf[Script=Greek]{Liberation Serif}

\usepackage{polyglossia}
\setmainlanguage{greek}

\usepackage{graphicx}
\graphicspath{{Слики/}}

\usepackage{amsmath}
\usepackage{amsfonts}
\usepackage{subfigure}
\usepackage{array}
\usepackage{amssymb}

\usepackage{xcolor}
\usepackage{colortbl}

\usepackage{tikz}
\usepackage{multimedia}
\usepackage{media9}

\setbeamertemplate{navigation symbols}{}
\setbeamerfont{footline}{size=\fontsize{14}{14}}
\setbeamertemplate{footline}{\insertframenumber/\inserttotalframenumber}

\addto\captionsgreek{\renewcommand\chaptername{Глава}}
\addto\captionsgreek{\renewcommand\contentsname{Содржина}}
\addto\captionsgreek{\renewcommand\listfigurename{Листа на слики}}
\addto\captionsgreek{\renewcommand\listtablename{Листа на табели}}
\addto\captionsgreek{\renewcommand\figurename{Слика}}
\addto\captionsgreek{\renewcommand\tablename{Табела}}
\addto\captionsgreek{\renewcommand\indexname{Индекс}}
\addto\captionsgreek{\renewcommand\bibname{Референци}}
\addto\captionsgreek{\renewcommand\abstractname{Апстракт}}

\newcolumntype{x}[1]{%
>{\centering\hspace{0pt}}m{#1}}%

\title{Воведна презентација - Машинско учење}
\author{Стефан Златинов}
\institute{Институт за Автоматика и Системско Инженерство 
\\ 
Факултет за електротехника и информациски технологии, Скопје}
\date{}


\begin{document}


\begin{frame}
\maketitle
\end{frame}


\begin{frame}
\frametitle{Клуч за предметот Машинско учење на е-курсеви}

\begin{itemize}

\item КСИАР 

\item со ГОЛЕМИ букви на кирилица

\item за сите предмети под АСИ

\end{itemize}

\end{frame}


\begin{frame}
\frametitle{Распределба на часови}

\begin{itemize}

\item 2 Предавања

\item 2 Аудиториски

\item 1 Лабораториски

\end{itemize}

\end{frame}


\begin{frame}
\frametitle{Начин на предавање}

\begin{itemize}

\item Аудиториски

\item Лабораториски

\end{itemize}

\end{frame}


\begin{frame}
\frametitle{Сакам да прашам}

\begin{itemize}

\item На час

\item На форум - на е-курсеви

\item На консултации

\item Користете кирилица!

\end{itemize}


\end{frame}


\begin{frame}
\frametitle{Не сакам}

Мразам / не сакам / не знам / не сакам да научам / не ме бива за:
\begin{itemize}

\item Математика

\item Програмирање

\item Англиски?

\end{itemize}


\end{frame}



\begin{frame}
\frametitle{Анкета}

\begin{itemize}

\item Колку имаат положено ПИА1?

\item Колку од вас имаат работено Пајтон?

\item Колку од вас имаат работено со git?

\item Дали ви одговараат моменталните часови?

\item Колку од вас се од КСИАР II?

\item Колку од вас се од КСИАР III?

\item Колку од вас се од КСИАР IV?

\item Колку од вас се од некој друг смер?

\item Планирате ли да присуствувате на час?

\end{itemize}

\end{frame}


\begin{frame}
\frametitle{Пајтон! - Зошто?}

\begin{itemize}

\item Лесен

\item Генерален јазик

\item Брзо се развива прототип

\item Затоа најчесто се користи во науката

\end{itemize}

\end{frame}


\begin{frame}
\frametitle{Стил на кодирање}

\begin{itemize}

\item Англиски јазик за с\`е освен

\item Македонски за интеракција со корисник

\item Почитувајте конвенции - PEP8

\item Развојна околина Pycharm community edition од JetBrains

\item Git - софтвер за верзионирање на код

\item Github - тука ќе го прикачуваме кодот

\end{itemize}

\end{frame}

\end{document}
